\documentclass{article}
\usepackage{graphicx}
\usepackage{amsmath}
\usepackage{amssymb}
\usepackage{enumerate}
\usepackage{xfrac}
\usepackage{tikz}
\usepackage{hyperref}

\usepackage[a4paper,margin=0.3in]{geometry} % Required for inserting images
\setlength{\parindent}{0pt}
\pagenumbering{gobble}

\title{time_series_cheat_sheet}
\author{Vasudev Joy}
\date{April 2025}

\begin{document}

\section*{Time Series Cheat Sheet}

\subsection*{Chapter 2: Basics}

\begin{enumerate}
    \item \textbf{Stationarity:} A discrete time stochastic process is called strongly stationary if the joint cdf of \(\{X_{t_1}, X_{t_2}, \dots,X_{t_n}\}\) is the same as the joint cdf of \(\{X_{t_1 + s}, X_{t_2+s}, \dots,X_{t_n+s}\}\). Weak stationarity means that the joint moments of order 1 and 2 of \(\{X_{t_1}, X_{t_2}, \dots,X_{t_n}\}\) exist and are finite and are equal to the the joint moments for \(\{X_{t_1 + s}, X_{t_2+s}, \dots,X_{t_n+s}\}\) for any valid \(s\). Equivalently, we can say a process is second order stationary if the mean is constant and if the covariance is only dependent on lag \(\tau\).

    \item \textbf{White Noise:} A process which has constant mean, variance and no autocovariance.

    \item \textbf{GLP:} A general linear process is one that depends only on the present and past and is given by
    \[
    X_t = \sum_{k=0}^{\infty} g_k \epsilon_{t-k} = \left(\sum_{k=0}^{\infty} g_k B^{k}\right)\epsilon_{t} 
    \]
    where \(\epsilon_t\) is a white noise process and it can be easily shown that this process is stationary and the autocovariance sequence of a GLP is given by \(s_\tau = \sigma_\epsilon^2 \sum_{k=0}^{\infty}g_k g_{k+\tau}\).

    \item \textbf{MA(q):} A moving average process given by
    \[
    X_t = \mu - \theta_{0,q}\epsilon_t - \theta_{1,q}\epsilon_{t-1} - \dots - \theta_{q,q}\epsilon_{t-q} = \mu - \sum_{i=0}^{q} \theta_{i,q} \epsilon_{t-i} = \mu +\Theta(B)\epsilon_t
    \]
     and 
    \[
    \Theta(B) 
    = 1-\theta_{1,q}\epsilon_{t-1}B^1 - \dots - \theta_{q,q}\epsilon_{t-q}B^q 
    = 1 - \sum_{i=1}^{p}\theta_{i,p}B^i
    \]
    is known as the characteristic polynomial. A moving average process is always stationary because it can be trivially written as a GLP. Furthermore, it is invertible if roots of \(\Theta(z)\) outside \(|z|\leq 1\).
    \item \textbf{AR(p):} The autoregressive process is given by
    \[
    X_t = \epsilon_t +\phi_{1,p}X_{t-1} + \dots + \phi_{p,p}X_{t-p} 
    \quad \text{ or } \quad 
    \Phi(B)X_t = \epsilon_t
    \]
    where
    \[
    \Phi(B) 
    = 1 - \phi_{1,p}B - \dots - \phi_{p,p}B^p
    = 1 - \sum_{i=1}^{p}\phi_{i,p}B^i
    \]
    is known as the associated polynomial. This process is automatically invertible and also it is stationary if the roots of \(\Phi(z)\) lie outside \(|z|\leq 1\).
    \item \textbf{ARMA(p,q)} is given by
    \[
    \Phi(B)X_t = \Theta(B)\epsilon_t
    \]
    and this is stationary or invertible if the roots of \(\Phi(z)\) or \(\Theta(z)\) lie outside \(|z|\leq 1\).

    \item \textbf{Trends and Seasonality:}

    \item \textbf{Common Tricks}:
    \begin{enumerate} [a.]
        \item Remembering that Cov\((\epsilon_t, \epsilon_{t'}) =0\) for \(t \neq t'\) to collapse the sum for calculation of autocovariance then setting changing the indexing.
        \item Using the geometric series forms for inverting polynomials. 
    \end{enumerate}
\end{enumerate}

\subsection*{Chapter 3: Spectral Analysis}

\begin{enumerate}
    \item \textbf{Spectral Representation Theorem:} For real valued discrete time \textbf{stationary} process, there exists an orthogonal process \(\{Z(f)\}\) on \([-1/2, 1/2]\) such that \(X_t = \int_{-1/2}^{1/2} \exp( 2\pi fti)\;dZ(f)\). This is like a fourier transform where \(Z(f)\) is the "fourier" transform of \(X_t\). The process has the following properties
    \begin{enumerate} [a.]
        \item \(E\{dZ(f)\} = 0\) for all \(|f|\leq 1/2\)
        \item \(E\{|dZ(f)|^2\} = dS^{(I)}(f)\) and \(S^{(I)}\) is the integrated spectrum of \(X_t\)
        \item For \(f \neq f' \in (-1/2, 1/2]\), Cov\((dZ(f), dZ(f')) = 0\)
    \end{enumerate}
    \item \textbf{Autocovariance:} We can relate the autocovariance sequence with the integrated spectrum
    \[
    s_\tau = \int_{-1/2}^{1/2} \exp( 2\pi fti)\;dS^{(I)}(f) = \int_{-1/2}^{1/2} \exp( 2\pi fti)S(f)\;df
    \]
    where the last equality holds only if the integrated spectrum is differentiable everywhere. Furthermore, we see that
    \[
    S(f) = \sum_{\tau = -\infty}^{\infty} s_{\tau} \exp(- 2\pi fti)
    \]
    and hence we see that \(S(f)\) is the fourier transform of \(s_\tau\) and \(s_\tau\) is the inverse fourier transform of \(S(f)\) and they form a fourier pair. An easy way to remember is that the fourier transform is in the frequency domain.
    \item \textbf{SDF}: Assuming the SDF exists it has the following properties
    \begin{enumerate}[a.]
        \item FTC type \(S^{(I)}(f) = \int_{-1/2}^{f}S(f')\; df'\)
        \item Integrated spectrum is monotonous
        \item SDF is symmetric
    \end{enumerate}
    \item \textbf{Classification of Spectra:} The integrated spectrum can be broken down into absolutely continuous and discrete parts where
    \[
    S^{(I)}(f) = S_{ac}^{(I)}(f) + S_{d}^{(I)}(f) \quad \text{ where } \quad  S_{ac}^{(I)}(f) = \int_{-1/2}^f S(f') \; df \text{ and  } S_{d}^{(I)}(f) \text{ is a step function}
    \]
    \item \textbf{Sampling and Aliasing:} Given a sample rate, we need to determine what range of frequencies to integrate over to get \(X(t)\). This leads to the defintion of the Nyquist frequency, \(f_\mathcal{N} = 1/2\Delta t\),  and we have the spectral representation theorems
    \[
    X_t = \int_{-f_\mathcal{N}}^{f_\mathcal{N}} e^{2\pi i ft \Delta t} \; dZ(f) \text{ and } X(t) = \int_{-\infty}^{\infty} e^{2\pi i ft}\;dZ(f)
    \]
    where often discrete time series are continuous time series sampled at a constant frequency \(\Delta t\). Because of this sampling we observe aliasing where The connection between discrete time and continuous time processes is that
    \[
    S_{X_t}(f) = \sum S_{X(t)}\left(f + \frac{k}{\Delta t} \right) \text{ for } |f|\leq 1/2\Delta t
    \]
    \item \textbf{Linear Filtering:} A linear filter is a function from one sequence to another sequence that has the following properties
    \begin{enumerate}[a.]
        \item \(L(a(x_n)) = aL((x_n))\)
        \item \(L((x_1)+(x_2)) = L(x_1) + L(x_2)\)
        \item \(L(B^\tau(x_n))= B^\tau L((x_n))\)
    \end{enumerate}
    Then we also have \(y_{f,t}= e^{i 2\pi f t}y_{f,0}\) and we set \(G(f) = y_{f,0}\) where \(|G(f)|\) is the gain and arg\((G(f))\) is the phase. Importantly, any linear filter can be represented as
    \[
    L(X_t) = \sum_{u=- \infty}^{\infty} g_uX_{t-u} \equiv Y_t \quad \text{ and }
    \]
    \[ 
    L(e^{i 2\pi ft}) 
    = \sum_{u = -\infty}^{\infty} g_u e^{i 2 \pi f (t -u)} 
    = e^{i 2 \pi f t}G(f) \implies G(f)
    = \sum_{u = -\infty}^{\infty} g_u e^{-i 2 \pi f u} \implies G(f) \leftrightarrow \{g_u\}
    \]
    We also have 
    \[
    dZ_Y(f)=G(f)dZ_X(f) \text{ and } E(|dZ_Y(f)|^2) = |G(f)|^2E(|dZ_X(f)|^2) \text{ and } S_Y(f) = |G(f)|^2S_X(f)
    \]
    if the spectral density function exists. The actual applications of this is by convolving the signal with the inverse fourier transform of a desired frequency response function we can filter out the undesired frequencies. To \textbf{find \(G(f)\)}, we can feed in a \(e^{i2 \pi ft}\) into the linear filter and the coefficient of \(e^{i2 \pi ft}\) would be the transfer/response function. We could also find the fourier transform of the impulse response sequence.

    \item \textbf{Spectral Density using Linear Filters:} By \textbf{passing through} \(L\{\xi_{f,t}\}\) to get \(G(f)\), we can convert the spectral density of \(X\) to the spectral density of \(Y\) and then subsequently inverse fourier tranform the spectral density function to the autocovariance sequence.
    \begin{enumerate} [a.] 
        \item \textbf{MA:} For \(X_t  = \epsilon_t - \theta_{1,q}\epsilon_{t-1} - ... - \theta_{q,q}\epsilon_{t-q}\), we have \(X_t = L(\epsilon_t) = \epsilon_t - \theta_{1,q}\epsilon_{t-1} - ... - \theta_{q,q}\epsilon_{t-q}\) and hence \(G(f) = \Theta(e^{-i 2 \pi f}) \) and \(S_X(f) = |\Theta(e^{i 2 \pi ft})|^2S_\epsilon(f) = |\Theta(e^{-i 2 \pi f})|^2\sigma_\epsilon^2\).
        \item \textbf{AR:} For \(X_t - \phi_{p,1}X_{t-1}- ...- \phi_{p,p}X_{t-p} = \epsilon_t\), we have \(L(X_t) = X_t - \phi_{p,1}X_{t-1}- ...- \phi_{p,p}X_{t-p} = \epsilon_t\) and \(G(f) = \phi(e^{-i 2 \pi f})\) and so \(|\Phi(e^{-i 2 \pi f})|^2 S_X(f) = S_\epsilon(f) = \sigma_\epsilon^2\)
    \end{enumerate}

    \item \textbf{Main Tricks:}
    \begin{enumerate} [a.]
        \item Remember that \(E(dZ^*(f)dZ(f)) = S(f)df\).
        \item Remember that \(S(f)\) is the fourier transform of \(s_\tau\) so we can calculate the spectral density by first calculating the autocovariance sequence.
        \item Remember the folding trick.
        \item For ARMA, Linear filtering, remember that the trick is to get \(\Phi(B)X_t = Y_t = \Theta(B)\epsilon_t\) then use lienar filtering to calculate the frequency response function.
    \end{enumerate}

\end{enumerate}
\subsection*{Chapter 4: Estimation}

\begin{enumerate} 
    \item \(\textbf{MSE} = E((\hat\theta - \theta)^2) = Var(\hat\theta) + (bias(\hat\theta))^2\) and if MSE goes to 0, we have that the variance and bias goes to 0. 
    
    \item \textbf{Sample Mean}: converges in MSE to true mean when the autocovariance sequence is absolutely summable.
    
    \item \textbf{Autocovariance:} A natural estimator for autocovariance is 
    \[
    \hat s_\tau^{(u)} = \frac{1}{N - |\tau|}\sum_{t = 1}^{N - |\tau|} (x_t- \bar x)(x_{t+|\tau|} - \bar x)
    \]
    which is unbiased when when we do not have to estimate \(\bar x = \mu.\) However, this might not be the best estimator and a second candidate is
    \[
    \hat s_\tau^{(p)} = \frac{1}{N }\sum_{t = 1}^{N - |\tau|} (x_t- \bar x)(x_{t+|\tau|} - \bar x)
    \]
    and this is often preferred because the \textbf{MSE is lower}, the \textbf{autocovariance goes to 0} as \(\tau \to N-1\) and because \(\hat s_t^{(p)}\) is a \textbf{positive semidefinite} sequence whereas \(\hat s_t^{(u)}\) might not be.
    
    \item \textbf{Spectral Density Estimation:} For spectral density estimation we just do the fourier transform of the autocovariance sequence estimate to get
    \[
    \hat S^{(p)}(f) = \sum_{\tau 
    = -(N-1)}^{N-1} \hat s_\tau^{(p)} e^{-2i\pi f t} 
    = \frac{1}{N}\left| \sum_{t=1}^N X_t  e^{-2i\pi f t} \right|^2
    \]
    where this is just the normalised square of the fourier transform of the time series. This is known as the periodogram/ power spectrum. However, this is a good approximation for some processes and bad for others, its variance does not go to 0 as \(N \to \infty\) and lastly, Cov\((\hat S^{(p)}(f), \hat S^{(p)}(f') \approx 0\) if \(f, f'\) are fourier frequencies.

    \item \textbf{Expected Value of Periodogram}: By using the fact that the estimator \(\hat S^{(p)}(f)\) is the normalised square of the fourier transform of the time series, we see that the expected value of the periodogram is given by
    \[
    E(\hat S^{(p)}(f)) = \int_{-1/2}^{1/2} \mathcal{F}(f - f')S(f')df' = \mathcal{F}*S \text{ where } \mathcal{F} = \frac{\sin^2(N \pi f)}{N \sin^2(\pi f)}
    \]
    Because the convolution of another function with a delta function recovers the original function, we want the Fejers kernel to be close to a dirac delta function. So the Fejers kernel approaches \(N\) as \(f \to 0\), for any frequency not 0, \(\mathcal{F}(f) \to 0\) pointwise, it achieves its maximum at 0, for any fourier frequency, \(f_k\), \(\mathcal{F}(f_k) = 0\) and integrates to 1 on \([-1/2, 1/2]\). Therefore, as \(N \to \infty\), \(\mathcal{F}\to \delta\) and \(E(\hat S^{(p)}(f)) \to S(f)\), ie asymptotic unbiased.

    \item \textbf{Bias tapering:} The effects of side lobe leakage is worse for data with large frequency domain. A data taper is a sequence of the length N such that \(\sum_i h_i^2 = 1\). We can the fourier transform \(h_tX_t\) and get
    \[
    \hat S^{(d)}(f) = \left| \sum_{t=1}^N h_tX_t  e^{-2i\pi f t} \right|^2 \text{ and } E(\hat S^{(d)}(f)) = \int_{-1/2}^{1/2} \mathcal{H}(f - f')S(f')df' \text{ where } \mathcal{H}(f) = \left|\sum_{t=1}^{N}h_te^{-i2 \pi ft}\right|^2
    \]
    Ideally we want to choose \(h_t\) such that \(\mathcal{H}\) is close to a delta function and also if \(h_t = 1/\sqrt{n}\) then \(\hat S^{(d)}(f)  = \hat S^{(p)}(f) \). The downfall is that the resolution of the spectral estimate is worse (ie. wider central lobe).

    \item \textbf{Common Tricks:}
    \begin{enumerate} [a.]
        \item Remember the sum order swapping from diagonal sums to row sums.
        \item Comparing coefficients to recover the original characteristic polynomial.
    \end{enumerate}
    \end{enumerate}

    

\subsection*{Chapter 5: Model Fitting (AR Models)}

\begin{enumerate}
    \item \textbf{Yule Walker:} The Yule Walker method leverages that fact that
    \[
    s_\tau = \sum_{j=1}^p \phi_{j,p}\;s_{\tau-j} \quad \text{ and } \quad     \hat \sigma_\epsilon^2 = \hat s_0 - \sum_{j=1}^{p} \hat \phi_{j,p} \hat s_j
    \]
    so that we have \(\hat\phi_p=\hat\Gamma^{-1}\hat\gamma_p\) where \(\hat\Gamma^{-1}\) is the estimated Toeplitz matrix and \(\hat\gamma_p\) is the estimated autocovariance sequence. Furthermore, using the formulation for SDFs based on linear filtering we have
    \[
    \hat S(f) = \frac{\hat \sigma_\epsilon^2}{\left|\hat\Phi(e^{i 2 \pi ft})\right|^2}
    \]
    And Yule Walker always gives a stationary sequence if using the biased estimator \(\hat s^{(p)}\).

    \item \textbf{LS:} Just the usual least squares where the covariates are the past p time points and the dependent is the current time point. We can find by minimising sum of squared errors to get
    \[
    \hat \phi = (F^TF)^{-1}F^TX \quad \text{ and } \quad \hat \sigma_\epsilon^2 = \frac{(X- F\hat \phi)^T(X - F \hat \phi)}{N-2p}
    \]
    Additionally, if you append 0s behind to \(1-p\) and 0s ahead to \(N+p\) and the calculate the LS estimate, it is equivalent to the Yule-Walker estimate.
    
    \item \textbf{MLE:} This is equivalent to LS approach when we assume the process is Gaussian. But can be maximised by using
    \[
    f(X_1, X_2, \dots, X_n|\phi, \sigma_\epsilon^2 ) = f(X_1, X_2, \dots, X_n)\prod_{t=p+1}^N f(X_t|X_{t-1}, X_{t-2}, \dots, X_{t-p}, \phi, \sigma_\epsilon^2)
    \]

    \item \textbf{Model Selection:} You choose the model which has the least number of parameters but which is still effective and the criterion is given by
    \[
    \text{AIC} = 2k -2\ln(\ell (\hat \phi, \sigma_\epsilon^2))
    \]
    where \(k = p+1\) and \(\ell(\cdot, \cdot)\) is the liklihood function. 
\end{enumerate}

\subsection*{Chapter 6: Forecasting}

\begin{enumerate} [a.]
    \item \textbf{Formulation:} For a stationary and invertible ARMA process (because we need GLP) we have \(X_{t+l} = \Psi(B)\epsilon_{t+l}\) where \(\Psi\) is the GLP. Next clip the part dependent on future because the expected value of their contributions are 0 to get
    \[
    X_t(l) = \sum_{k=0}^{\infty} \psi_{k+l}\epsilon_{t-k} = \Psi^{(l)}(B)\epsilon_t
    \]
    This forecast minimises the expected squared difference between the forecast and the actual representation. We also have the l-step Ahead Prediction Variance
    \[
    \sigma^2(l)= E((X_{t+l}-X_t(l))^2) = \sigma_\epsilon^2\sum_{k=0}^{l-1} \psi_k^2
    \]
    \item \textbf{Inverting:} Assuming \(X_t\) is also invertible we see
    \[
    \epsilon_t = \Psi^{-1}(B)X_t \implies X_t(l) = \Psi^{(l)}(B)\Psi^{-1}(B)X_t = G^{(l)}(B)X_t
    \]
    \item \textbf{Error:}
    \[
    e_t(l) = X_{t+l} - X_t(l) = \sum_{k=0}^{l-1} \psi_{k}\epsilon_{t+l-k} \quad \text{and} \quad \text{Cov}(e_t(l), e_t(m+l))=\sigma_\epsilon^2\sum_{k=1}^{l-1}\psi_k\psi_{k+m}
    \]
    \item \textbf{Update:}
    \[
    X_{t+1} (l)= X_t(l+1)+\psi_l(X_{t+1} - X_t(1))
    \]
\end{enumerate}

\subsection*{Chapter 7: Bivariate}

\begin{enumerate} [a.]
    \item \textbf{Joint Stationarity:} Two real valued discrete stochastic processes are jointly stationary stochastic process if both \(X_t, Y_t\) are each weakly stationary and cross covariance is dependent only on lag \(\tau\).
    \item \textbf{Cross Covariance:} The first is at time \(t\) and the second is at time \(t + \tau\)
    \[
    s_{X, Y, \tau} = E((X_t-\mu_X)(Y_{t+\tau} - \mu_Y))
    \]
    Furthermore, not necessarily \(s_{X, Y, \tau} = s_{X, Y, -\tau}\) and the sequence is generally assymetric.
    \item \textbf{Estimating CCV:} We can estimate this similar to the biased estimator \(\hat s_\tau^{(p)}\) from before

    \item \textbf{Linear Filtering w Noise Example:} For 0 mean stationary  \(X_t, Y_t\) where \(Y_t = \sum_{u = -k}^{u=k}g_uX_{t-u} + \eta_t\) we have a useful way to get the cross covariance sequence
    \[
    s_{X, Y, \tau} = E\left(X_t\left(\sum_{u = -k}^{u=k}g_uX_{t-u} + \eta_t \right)\right) = \sum_{u = -k}^{u=k}g_u s_{X, \tau-u} = g*s_X
    \]
    \item \textbf{Cross Spectra:} Just the fourier transform of the cross covariance sequence and also no longer real value because ccv sequence is not symmetric so complex parts do not cancel.
    \[
    S_{X, Y}(f) = \sum_{\tau = -\infty}^{\infty} s_{X,Y,\tau} \;e^{-i2\pi f\tau} \text{ for } |f|\leq 1/2
    \]
    We do have \(S_{X, Y}^*(f) = S_{X, Y}(-f)\) ie skew symmetric. Furthermore, if the two processes are \textbf{cross orthogonal}, we also have that 
    \[
    E(dZ_X(f)dZ_Y(f')) = \begin{cases}
        S_{X,Y}(f)\;df & \text{ if }f=f'\\
        0 & \text{ otherwise}
    \end{cases}
    \]
    and hence we see
    \[
    s_{X,Y, \tau} = E(X^*_tY_{t+\tau}) = \int_{-1/2}^{1/2}\int_{-1/2}^{1/2}e^{-i 2 \pi f t}e^{i 2 \pi f' (t + \tau)} E(dZ_X(f)dZ_Y(f')) = \int_{-1/2}^{1/2} e^{-i2\pi f\tau} S_{X,Y}(f)\;df
    \]
    and we have the \textbf{spectral matrix} which is Hermitian
    \[
    S(f) = \begin{pmatrix}
        S_X(f)&S_{X,Y}(f)\\
        S_{Y,X}(f) & S_Y(f)
    \end{pmatrix}
    \]

    \item \textbf{ Coherence:} Kinda like \(R^2\) and is given by
    \[
    \gamma^2_{X,Y}(f) = \frac{|S_{X,Y}(f)|^2}{S_X(f)S_Y(f)}
    \]

    \item \textbf{Linear Filter + Noise Model:} Like above we also have
    \[
    S_{X,Y}(f) = G(f) S_X(f) \text{ and } S_Y(f) = |G(f)|^2S_X(f) + S_\eta(f)
    \]

    \item \textbf{Bivariate AR}: We get
    \[
    \Phi(B)X_t = \epsilon_t \text{ where } \Phi(\textbf{z}) = 1 - \phi_{1,p }z - \dots - \phi_{p,p}x^p \text{ where } \phi_{i,p} \in \mathbb{R}^{2 \times 2}
    \]
    and \(\epsilon_t\) is a bivariate white noise process and \(E(\epsilon_t^Te_s) = \Sigma\) if \(s = t\) and this is stationary if roots of determinant of \(\phi(\textbf{z})\) are outside 1.

\end{enumerate}
     


\end{document}